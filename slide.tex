%!TEX TS-program = xelatex
\documentclass[aspectratio=169]{beamer}
\usetheme{hust}

\usepackage{qrcode}
\usepackage{graphicx}
\usepackage{xcolor}
\usepackage[absolute,overlay]{textpos}
\usepackage{tikz}
\usepackage{calc}
\usepackage{polyglossia}
\setmainlanguage{english}
\usepackage{amsthm,amsmath,amssymb}
\newtheorem{thm}{Theory}
\newtheorem{defn}{Definition}
\usepackage{subcaption}
\usepackage{dutchcal}
\usepackage{url}

%%%%%%%%%%%%%%%%%%%%%%%%%%%%%%%%%%%%%%%%%%%%%%%%%%%%%%%%%%%%%%%%%%%%%%
% LaTeX Overlay Generator - Annotated Figures v0.0.1
% Created with http://ff.cx/latex-overlay-generator/
%%%%%%%%%%%%%%%%%%%%%%%%%%%%%%%%%%%%%%%%%%%%%%%%%%%%%%%%%%%%%%%%%%%%%%
%\annotatedFigureBoxCustom{bottom-left}{top-right}{label}{label-position}{box-color}{label-color}{border-color}{text-color}
\newcommand*\annotatedFigureBoxCustom[8]{\draw[#5,thick,rounded corners] (#1) rectangle (#2);\node at (#4) [fill=#6,thick,shape=circle,draw=#7,inner sep=2pt,font=\sffamily,text=#8] {\textbf{#3}};}
%\annotatedFigureBox{bottom-left}{top-right}{label}{label-position}
\newcommand*\annotatedFigureBox[4]{\annotatedFigureBoxCustom{#1}{#2}{#3}{#4}{gray}{gray}{black}{black}}
\newcommand*\annotatedFigureText[4]{\node[draw=none, anchor=south west, text=#2, inner sep=0, text width=#3\linewidth,font=\sffamily] at (#1){#4};}
\newenvironment {annotatedFigure}[1]{\centering\begin{tikzpicture}
\node[anchor=south west,inner sep=0] (image) at (0,0) { #1};\begin{scope}[x={(image.south east)},y={(image.north west)}]}{\end{scope}\end{tikzpicture}}
%%%%%%%%%%%%%%%%%%%%%%%%%%%%%%%%%%%%%%%%%%%%%%%%%%%%%%%%%%%%%%%%%%%%%%

\DeclareGraphicsExtensions{.eps, .pdf, .png, .jpeg, .jpg}
\graphicspath{{figures/}{diagrams/}}

\setbeameroption{show notes} % show notes and slides
% \setbeameroption{hide notes} % Only slides
% \setbeameroption{show only notes} % show only notes
\addtobeamertemplate{note page}{}{\thispdfpagelabel{notes:\insertframenumber}}
\setbeamertemplate{note page}[plain]
\setbeamertemplate{theorems}[numbered]
\setbeamertemplate{caption}[numbered]

\newcommand{\Image}[1]{%
    \sbox0{\includegraphics[height=0.65\paperheight]{#1}}%
    \ifdim\wd0 < \textwidth
    \includegraphics[height=0.65\paperheight]{#1}%
  \else
  \includegraphics[width=\textwidth]{#1}%
  \fi%
}

\AtBeginSubsection[]
{
  \begin{frame}{Table of contents}
    \tableofcontents[currentsection, subsectionstyle=show/shaded/hide, subsubsectionstyle=show/show/show/hide]
  \end{frame}
}

% \titlegraphic{\includegraphics[height=\logoheight]{figures/sami-v2.pdf}}
\title{Graduation Thesis}
\subtitle{Grammatical Error Correction Using Machine Learning\\Web Application (GecWeb)}

\author{Student: Ngô Văn Cảnh - 20193204 \protect\linebreak Advisor: Prof. Dương Tấn Nghĩa}

\date{February 2025}

\begin{document}
\begin{frame}[noframenumbering,Title]
  \maketitle
\end{frame}

\note{
  Greatings, everyone.

  I am Ngô Văn Cảnh from Hanoi University of Science and Technology, class ET-E4 Elitech Program.

  Today, I am going to present my graduation thesis, which is about Grammatical Error Correction using machine learning web application, or GecWeb for short.

  Under the guidance of Professor Dương Tấn Nghĩa.
}

\begin{frame}{Table of Contents}
  \tableofcontents
\end{frame}

\note{
  In today's talks, I will cover the following topics:
}

\section{Introduction}

\subsection{Motivation}

\begin{frame}{Motivation}
  \begin{figure}
    \begin{center}
      \includegraphics[width=\textwidth]{figures/ef-epi-2024-english-crop.pdf}
      \begin{textblock*}{8cm}(\paperwidth-9cm, \paperheight-2.5cm)  % (x,y) coordinates from top-left
        \textbf{\Large $> 1.4$ billion speakers}

        \textbf{\Large $\sim 75\%$ non-native}~\cite{Ethnologue-2024}
      \end{textblock*}
    \end{center}
    \caption{English Proficiency bands by countries \citeyear{ef-epi-2024}}\label{fig:ef-epi}
  \end{figure}
\end{frame}

\note{
  English is one of the most widely used languages globally, spoken by approximately more than 1.4 billion speakers, with almost 75\% of them being non-native speakers

  As the number of english-as-a-second-language (esl) and english-as-a-foreign-language (efl) learners continues to grow, the need for effective language learning tools and resources has increased significantly.

  However, grammatical and spelling errors remain common challenges for many writers, affecting clarity and professionalism.
}

\subsection{Definition}

\begin{frame}{Definition of Grammatical Error Correction}
  \centering
  {\Huge
    \textcolor{red}{G}rammatical
    \textcolor{red}{E}rror
    \textcolor{red}{C}orrection
  }

  \vfill

  {\Large
    He \textcolor{red}{go} to the store and \textcolor{red}{buyed} some \textcolor{red}{apple's}.
  }

  \noindent\rule[0.5ex]{\linewidth}{1pt}

  {\Large
    He \textcolor{green}{goes} to the store and \textcolor{green}{bought} some \textcolor{green}{apples}.
  }

  \vfill
\end{frame}

\note[itemize]{
  \item Grammatical Error Correction, or GEC, is the task of automatically detecting and correcting errors in text.

  \item Despite its name, the task is not limited to grammatical errors, such as missing prepositions and mismatched subject-verb agreement but it also expect to fix orthographic and semantic errors, such as misspellings and word choice errors.

  \item The term Grammatical Error Correction is thus a bit misleading but is nevertheless now commonly understood to encompass errors that are not always strictly grammatical in nature.

  \item A more descriptive term would be Language Error Correction.
}

\subsection{Usage barriers}

\begin{frame}{Usage barriers}
  \begin{itemize}
    \item Command-line interface
    \item Requires capable hardware
  \end{itemize}

  \pause

  \vspace{1cm}

  {\Large $\Rightarrow$ Solution: A lightweight web interface}
\end{frame}

\note[itemize]{
  \item Despite significant recent advancements in GEC technology, many state of the art systems remain inaccessible to the general public due to their reliance on command line interfaces and high performance computing resources.

  \item This creates a barrier for non technical users, particularly those in developing countries with limited access to advanced technology and slow internet connections.

  \item This is where GecWeb comes in, it provides a lightweight, user friendly GEC system that can be easily
  accessed via mobile devices or low-end computers.
}

\section{Implementation}

\subsection{Architecture design}

\begin{frame}{Architecture design}
  Architecture design
\end{frame}

\subsection{System Design and Implementation}

\begin{frame}{System Design and Implementation}
  System Design and Implementation
\end{frame}

\section{Demonstration}

\subsection{Interface Overview}

\begin{frame}{Interface Overview}
  Interface Overview
\end{frame}

\section{Conclusion}

\subsection{Achievements}
\begin{frame}{Achievements}
  \begin{itemize}
    \item A web-based application for GEC that can be easily access by general public.
    \item Designed with accessibility in mind.
    \item Flexible to host different type of GEC models and system combination methods.
    \item Designed to be modular and extensible.
  \end{itemize}
\end{frame}

\subsection{Future work}
\begin{frame}{Future work}
  \begin{itemize}
    \item Deploy the system to a more realiable server.
    \item Scale the system to handle more users.
  \end{itemize}
\end{frame}

\note{
  Although archived the initial goal of creating a web-based application for GEC, GecWeb is no where near production ready.

  The backend is currently running on a low-end server, which is not enough to run all 3 models and prune to crash(which it sometimes did).

  I did not take care of the security aspect of the application, such as input validation, output encoding, et
  cetera.

  As well as the performance aspect, such as caching, load balancing, et cetera.
}


\begin{frame}{References}
  % \nocite{*} % Show all references in the cite.bib file even if they are not cited
  \bibliographystyle{plain}
  \bibliography{cite}
\end{frame}

\begin{frame}{~}
  \begin{center}
    \Large \color{hustred}{Thank you for your attention!}
  \end{center}
\end{frame}

\end{document}
