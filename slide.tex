%!TEX TS-program = xelatex
\documentclass[aspectratio=169]{beamer}
\usepackage{tikz}
\usepackage{calc}
\usepackage{polyglossia}
\setmainlanguage{vietnamese}
\usepackage{amsthm,amsmath,amssymb}
\newtheorem{thm}{Định lý}
\newtheorem{defn}{Định nghĩa}
\usepackage{subcaption}
\setbeamertemplate{theorems}[numbered]

%% TYPESET
\makeatletter
\newcommand{\leqnomode}{\tagsleft@true}
\newcommand{\reqnomode}{\tagsleft@false}
\makeatother
\let\Mathcal\mathcal
\usepackage{dutchcal}
\newcommand{\norm}[1]{\left\lVert#1\right\rVert}
\newcommand{\abs}[1]{\left\lvert#1\right\rvert}
\newcommand{\gr}{\operatorname{gr}}
\newcommand{\rank}{\operatorname{rank}}
\newcommand{\toto}{\rightrightarrows}%
\newcommand{\F}{\Mathcal{F}}%
\newcommand{\K}{\mathbb{K}}%
\newcommand{\R}{\mathbb{R}}%
\def\C{\mathbb{C}}%
\newcommand{\dom}{\operatorname{dom}}%
% \newcommand{\ker}{\operatorname{ker}}% already defined
\newcommand{\im}{\operatorname{im}}%
\newcommand{\g}{\mathcal{g}}%
\global\long\def\MP{\text{mp}}%

\newcommand{\Image}[1]{%
    \sbox0{\includegraphics[height=0.65\paperheight]{#1}}%
    \ifdim\wd0 < \textwidth
    \includegraphics[height=0.65\paperheight]{#1}%
  \else
  \includegraphics[width=\textwidth]{#1}%
  \fi%
}

\newif\iffirsttoc
\firsttoctrue
\AtBeginSubsection[]
{
    \begin{frame}
        \frametitle{Mục lục}  
        \iffirsttoc
            \tableofcontents
            \global\firsttocfalse
        \else
            \tableofcontents[currentsection,
            % hideothersubsections,
            subsectionstyle=show/shaded/hide,
            subsubsectionstyle=show/show/show/hide
            ]
        \fi
    \end{frame} 
}
\setbeamertemplate{caption}[numbered]

\usetheme{hust}
% \titlegraphic{\includegraphics[height=\logoheight]{figures/sami-v2.pdf}}
\title{Seminar II:}
% \subtitle{\textbf{\LARGE Tìm bán kính điều khiển có cấu trúc của hệ chịu\\đa nhiễu sử dụng các toán tử tuyến tính đa trị}}
\subtitle{\textbf{Ứng dụng toán tử tuyến tính đa trị trong việc tìm\\bán kính điều khiển có cấu trúc của hệ chịu đa nhiễu}}

\author{Nguyễn Đức Hùng--20212498M\\Phùng Anh Hùng--20212497M\\Nguyễn Đức Anh--20211315M}
\date{Tháng 10, 2022}

\begin{document}
\begin{frame}[noframenumbering,Title]
  \maketitle
\end{frame}

\section{Giới thiệu}
% \begin{frame}{Slide nháp}
% Nội dung cần nói
% \begin{itemize}
%     \item Giới thiệu chung về các định nghĩa
%     \item Hệ 
%     \item Tính điều khiển được
%     \item Bán kính điều khiển
% \end{itemize}
% {\large khả năng cao phần này sẽ gộp với phần 2, do cùng là giới thiệu các định nghĩa khái niệm}
% \end{frame}
\subsection{Giới thiệu}
\begin{frame}{Hệ điều khiển được}
    \begin{itemize}
        \item Hệ điều khiển tuyến tính:
        \begin{align}
            \dot{x}=Ax+Bu,x\in\mathbb{K}^{n},A\in\mathbb{K}^{n\times n},B\in\mathbb{K}^{n\times m},
        \end{align}
        \item Hệ là điều khiển được
        \begin{align}
            \rank\left[A\mid B\right]=n,
        \end{align}
        Trong đó
        \begin{equation}
        \left[A\mid B\right]=\left[\begin{array}{cccc}
        B & AB & \cdots & A^{n-1}B\end{array}\right].    
        \end{equation}
    \end{itemize}
\end{frame}
\begin{frame}{Bán kính điều khiển được}
    \begin{itemize}
        \item Chịu nhiễu
        \begin{equation}
            \left[A,B\right]\leadsto\left[\tilde{A},\tilde{B}\right]=\left[A,B\right]+\left[\Delta_{1},\Delta_{2}\right].
        \end{equation}
        \item Khoảng cách đến trạng thái không điều khiển được:
        \begin{align}
            r_{\mathbb{K}}\left(A,B\right)= & \inf\{\norm{\Delta_{1},\Delta_{2}}:\left[\Delta_{1},\Delta_{2}\right]\in\mathbb{K}^{n\times\left(n+m\right)} ,\nonumber \\
             &\left[A,B\right]+\left[\Delta_{1},\Delta_{2}\right]\text{ là không điều khiển được\}}.
        \end{align}
        % \item Công thức Eising:
        % \begin{align}
        %     r_{\mathbb{C}}\left(A,B\right)=\inf_{\lambda\in\mathbb{C}}\sigma_{\min}\left(\left[A-\lambda IB\right]\right).
        % \end{align}
        % \item Điều kiện Hautus:
        % \begin{align}
        %     \left(A,B\right) & \in\mathbb{K}^{n\times n}\times\mathbb{K}^{n\times m}\text{ là điều khiển được} \nonumber\\
        %      & \iff\rank\left[A-\lambda i,B\right]=n,\text{ }\forall\lambda\in\mathbb{C}
        % \end{align}
    \end{itemize}
\end{frame}

% \begin{frame}{Nhiễu của cặp ma trận}
%     \begin{itemize}
%         \item Cặp $\left(A,B\right)$ bị nhiễu affine
%         \begin{align}
%             \left[A,B\right]\rightsquigarrow\left[\tilde{A},\tilde{B}\right]=\left[A,B\right]+D\Delta E,\text{ }\Delta\in\mathcal{D},
%         \end{align}
%         với $D\in\mathbb{K}^{n\times l},E\in\mathbb{K}^{q\times\left(n+m\right)}$
% là ma trận cấu trúc cho trước và $\mathcal{D}\subset\mathbb{K}^{l\times q}$ là
% lớp nhiễu cho trước. 
%         \item Ma trận $A$ và $B$ chịu nhiều nhiễu
%         \begin{align}
%             \left[A,B\right]\rightsquigarrow\left[\tilde{A},\tilde{B}\right]=\left[A,B\right]+\sum_{i=1}^{N}D_{i}\Delta_{i}E_{i}.0
%         \end{align}
%     \end{itemize}
% \end{frame}
\subsection{Ánh xạ đa trị}
% \begin{frame}{Các khái niệm cần cover}
%     Slide này để nháp, không trong nội dung cuối
%     \begin{itemize}
%         \item Ma trận adjoint
%         \item Không gian đối ngẫu
%         \item Không gian bù trực giao của một không gian
%         \item Ánh xạ đa trị
%         \item Toán tử đa trị tuyến tính
%         \item chuẩn vector, chuẩn ánh xạ đa trị
%         \item ker và dom của ánh xạ đa trị
%     \end{itemize}
% \end{frame}

\begin{frame}{Một số định nghĩa và ký hiệu}
    \begin{itemize}
%         \item Không gian đối ngẫu
%         \begin{equation}
%             \left(\mathbb{K}^{n}\right)^{*}=\left\{ u^{*}\colon u\in\mathbb{K}^{n}\right\}
%         \end{equation}
%         \item Phần bù trực giao 
%         \begin{equation}
%             M^{\perp}=\left\{ u^{*}\in\mathbb{K}^{*}\colon u^{*}x=0\ \forall x\in M\right\} 
%         \end{equation}
%         \item Với $u^{*}\in\left(\mathbb{K}^{n}\right)^{*}$, $x\in\mathbb{K}^{n}$,
% ký hiệu 
%         \begin{equation}
%             u^{*}\left(x\right)=u^{*}x
%         \end{equation}
        \item Ánh xạ đa trị \(\F\)
        \begin{equation}
            \F\colon\K^{n}\toto\K^{m}
        \end{equation}
        \item Đồ thị của \(\F\)
        \begin{equation}
            \gr\F=\left\{ \left(x,y\right)\in\K^{n}\times\K^{m}\colon y\in\F\left(x\right)\right\} 
        \end{equation}
    \end{itemize}
\end{frame}

\begin{frame}{Một số định nghĩa và ký hiệu}
    \begin{itemize}
        \item Miền xác định của \(\F\)
        \begin{equation}
            \dom\F  =\left\{ x\in\K^{n}\colon\F\left(x\right)\ne\emptyset\right\}
        \end{equation}
        \item Nhân của \(\F\)
        \begin{equation}
            \ker\F  =\left\{ x\in\dom\F\colon0\in\F\left(x\right)\right\}
        \end{equation}
        \item Chuẩn của \(\F\)
        \begin{equation}
            \norm{\F}=\sup\left\{ \inf_{y\in\F\left(x\right)}\norm y\colon x\in\dom\F,\norm x=1\right\}
        \end{equation}
    \end{itemize}
\end{frame}

% \leqnomode
% \begin{frame}{Một số tính chất}

% \begin{align}
%   & y\in\F\left(x\right)\iff\F\left(x\right)=y+\F\left(0\right)  \\
%   & \vphantom{\frac11}\inf_{y\in\F\left(x\right)}\norm y\le\norm{\F}\norm x\quad\forall x\in\dom\F  \\
%   & \vphantom{\frac11}\norm{\F\left(x\right)}\le\norm{\F}\norm x\quad\forall x\in\dom\F \quad\text{nếu \(\F\) đơn trị}
% \end{align}
% \end{frame}
% \reqnomode
\section{Bán kính điều khiển có cấu trúc}
% \begin{frame}{Slide nháp}
% Nội dung cần nói
% \begin{itemize}
%     \item Định nghĩa của nhiễu affine, khái niệm ma trận cấu trúc
%     \item Định nghĩa của bán kính điều khiển
%     \item Đưa ra các theorem về công thức tính bán kính điều khiển
%     \item Dẫn dắt để nói cho tất cả các trường hợp
% \end{itemize}
% {\large Ưu tiên đưa các theorem về công thức trước, sau đó đưa các định nghĩa và công thức liên quan vào, vì các bổ đề có thể chỉ dùng để chứng minh chứ không cần thiết cho nội dung định lý}
% \end{frame}
\subsection{Bán kính điều khiển}
\begin{frame}{Định nghĩa}
    Nhiễu có cấu trúc
    \begin{equation}
        \left[{A},{B}\right] \leadsto{}\left[\widetilde{A},\widetilde{B}\right]=\left[A,B\right]+D\Delta E
    \end{equation}
    Bán kính điều khiển
    % Cho chuẩn $\norm{\cdot}$ trên $\K^{l\times q}$, một khoảng cách
% cấu trúc để một cặp $\left(A,B\right)$ không điều khiển được dưới
% nhiễu affine được định nghĩa bởi
    \begin{equation}
        r_{\K}^{D,E}\left(A,B\right)=\inf\left\{ \norm{\Delta}\colon\Delta\in\K^{l\times q},\left[\widetilde{A},\widetilde{B}\right]\text{ không điều khiển được}\right\} 
    \end{equation}
%     Nếu cặp $\left(A,B\right)$ điều khiển được dưới mọi loại nhiễu thì
% $r_{\K}^{D,E}\left(A,B\right)=\infty$.
%     Đại lượng $r_{\K}^{D,E}\left(A,B\right)$
% được gọi là bán kính điều khiển được có cấu trúc của hệ $\dot{x}=Ax+Bu$.
\end{frame}

\begin{frame}{Định nghĩa}
\begin{itemize}
    \item Ánh xạ đơn trị
    \begin{align}
    W_{\lambda}\colon & \K^{m+n}\to\K^{n} \nonumber\\
    W_{\lambda}\left(z\right) & =\left[\begin{array}{cc}
    A-\lambda I & B\end{array}\right]z,\quad \lambda\in\C.
    \end{align}
    \item Ánh xạ đa trị 
    \begin{align}
    EW_{\lambda}^{-1}D\colon&\K^{l}\toto\K^{q} \nonumber\\
    \left(EW_{\lambda}^{-1}D\right)\left(u\right)&=E\left(W_{\lambda}^{-1}\left(Du\right)\right).
    \end{align}
    \end{itemize}
\end{frame}



\begin{frame}{Định lý}
    Nếu $\mathbb{K}=\mathbb{C}$ thì:

    \begin{align}
    r_{\mathbb{C}}^{D,E}\left(A,B\right)=\frac{1}{\sup_{\lambda\in\mathbb{C}}\norm{EW_{\lambda}^{-1}D}}.\label{eq:3.3}
    \end{align}
\end{frame}

\begin{frame}{Nghịch đảo Moore-penrose tổng quát}
\begin{itemize}
    \item Cho \(G\in\C^{n\times p}\), \(\rank_{\text{row}}(G)=n\), \(\F_{G}(z)=Gz\)
\begin{equation}
d\left(0,\mathcal{F}_{G}^{-1}\left(y\right)\right)=\norm{G^{\dagger}y},\quad\norm{\mathcal{F}_{G}^{-1}}=\norm{G^{\dagger}}.\label{eq:3.5}
\end{equation}
    \item Cho \(G\in\C^{n\times p}\), \(\F_{G}\left(z\right)=Gz\)
\begin{equation}
\F_{G}^{\dagger}\left(y\right)=\begin{cases}
z\quad s.t.\quad Gz=y,\norm z=d\left(0,\F_{G}^{-1}\left(y\right)\right) & y\in\im\F_{G,}\\
\emptyset & y\notin\im\F_{G}.
\end{cases}\label{eq:3.6}
\end{equation}
\end{itemize}
\end{frame}

% \begin{frame}{Định nghĩa}
%     Cho toán tử tuyến tính $\F_{G}\left(z\right)=Gz$
% trong đó $G\in\C^{m\times p}$, nghịch đảo Moore-Penrose tổng quát
% $\F_{G}^{\dagger}$ của $\F_{G}$ được định nghĩa bởi:

% \end{frame}

\subsection{Ví dụ}
\begin{frame}{Ví dụ}
    \begin{align}
        \dot{x}&=Ax\left(t\right)+Bu\left(t\right), \\
        A&=\left[\begin{array}{cc}
        0 & 1\\
        1 & 0
        \end{array}\right],\\
        B &=\left[\begin{array}{c}
        1\\
        0
        \end{array}\right],\\
        \left[\begin{array}{ccc}
        0 & 1 & 1\\
        1 & 0 & 0
        \end{array}\right]&\leadsto\left[\begin{array}{ccc}
        \delta_{1} & \delta_{1}+1 & \delta_{2}+1\\
        \delta_{1}+1 & \delta_{1} & \delta_{2}
        \end{array}\right].
    \end{align}
\end{frame}

\begin{frame}{Ví dụ}
    \begin{align}
        \left[A,B\right]&\leadsto\left[A,B\right]+D\Delta E, \\
        D&=\left[\begin{array}{c}
        1\\
        1
        \end{array}\right],\\
        E&=\left[\begin{array}{ccc}
        1 & 1 & 0\\
        0 & 0 & 1
        \end{array}\right].
    \end{align}
\end{frame}


\begin{frame}{Ví dụ}
    \begin{align}
        \left(E\left[A-\lambda I,B\right]^{-1}D\right)\left(v\right)
        &= E\left[A-\lambda I,B\right]^{-1}\left[\begin{array}{c}
1\\ 1\end{array}\right]v \nonumber\\
        &= E\left[A-\lambda I,B\right]^{-1}\left[\begin{array}{c}
v\\
v\end{array}\right]
    \end{align}
\end{frame}

\begin{frame}{Ví dụ}
    \begin{align}
        \left[A-\lambda I,B\right]^{-1}\left[\begin{array}{c}
        v\\
        v
        \end{array}\right]&=\left\{ \left[\begin{array}{c}
        p\\
        q\\
        r
        \end{array}\right]\colon\left[A-\lambda I,B\right]\left[\begin{array}{c}
        p\\
        q\\
        r
        \end{array}\right]=\left[\begin{array}{c}
        v\\
        v
        \end{array}\right],\,\forall p,q,r\in\C\right\} \nonumber\\
        %%% 
        &=\left\{ \left[\begin{array}{c}
        p\\
        q\\
        r
        \end{array}\right]\colon\left[\begin{array}{c}
        q+r-\lambda p\\
        q-p\lambda
        \end{array}\right]=\left[\begin{array}{c}
        v\\
        v
        \end{array}\right],\,\forall p,q,r\in\C\right\} .
    \end{align}
\end{frame}

\begin{frame}{Ví dụ}
    \begin{align}
        \left(E\left[A-\lambda I,B\right]^{-1}D\right)\left(v\right)&=\left\{ E\left[\begin{array}{c}
        p\\
        q\\
        r
        \end{array}\right]\colon\left[\begin{array}{c}
        q+r-\lambda p\\
        q-p\lambda
        \end{array}\right]=\left[\begin{array}{c}
        v\\
        v
        \end{array}\right],\,\forall p,q,r\in\C\right\} \nonumber\\
        %%
        &=\left\{ \left[\begin{array}{c}
        p+q\\
        r
        \end{array}\right]\colon\left[\begin{array}{c}
        q+r-\lambda p\\
        q-p\lambda
        \end{array}\right]=\left[\begin{array}{c}
        v\\
        v
        \end{array}\right],\,\forall p,q,r\in\C\right\} \nonumber\\
        %%
        &=\left\{ \left[\begin{array}{c}
        q+v+q\lambda\\
        v-q+\lambda\left(v+q\lambda\right)
        \end{array}\right]:q\in\C\right\} .
    \end{align}
\end{frame}

\begin{frame}{Ví dụ}
    \begin{align}
        ax_{1}+x_{2}&=b\\
        x_{1}&=q+v+q\lambda,\\
        x_{2}&=v-q+\lambda\left(v+q\lambda\right).
    \end{align}
    \begin{itemize}
        \item \(q = q_{0}\), \(q = q_{1}\) \(\implies\) \(a = 1 - \lambda\), \(b = 2v.\)
    \end{itemize}
    \begin{align}
        \left(1-\lambda\right)x_{1}+x_{2}&=2v,\\    
        x_{1}&=q+v+q\lambda,\\
        x_{2}&=v-q+\lambda\left(v+q\lambda\right).
    \end{align}
\end{frame}

\begin{frame}{Ví dụ}
\begin{itemize}
    \item \(\lambda = -1\)
        \begin{align}
            x_1 &= v,\\
            x_2 &= 0,\\
            d\left(0,\left(E\left[A-\lambda I,B\right]^{-1}D\right)\left(v\right)\right)&=\abs{v}.
        \end{align}
    \item \(\lambda\ne-1\),  \(\left(\C,\norm{\cdot}_{\infty}\right)\)
        \begin{align}
            2\abs v&=\abs{\left(1-\lambda\right)x_{1}+x_{2}} \nonumber\\
            &\le\abs{\left(1-\lambda\right)x_{1}}+\abs{x_{2}}\nonumber\\
            &\le\left(\left|1-\lambda\right|+1\right)\max\left\{ \left|x_{1}\right|,\left|x_{2}\right|\right\} \nonumber\\
            &=\left(\left|1-\lambda\right|+1\right)\norm{\left[\begin{array}{c}
                x_{1}\\
                x_{2}
            \end{array}\right]}_{\infty}
        \end{align}
    \end{itemize}
\end{frame}

\begin{frame}{Ví dụ}
\begin{itemize}
    \item \(\lambda\ne-1\),  \(\left(\C,\norm{\cdot}_{\infty}\right)\)
        \begin{align}
        \frac{2\left|v\right|}{\left|\lambda-1\right|+1}&\le\norm{\left[\begin{array}{c}
        x_{1}\\
        x_{2}
        \end{array}\right]}_{\infty}.
        \end{align}
    \item Dấu bằng:
    \begin{align}
        x_{1}&=e^{i\varphi}x_{2},\\
        x_{2}&=\frac{2v}{1+\left|\lambda-1\right|}\\
        -\left(\lambda-1\right)e^{i\varphi}&=\left|\lambda-1\right|
    \end{align}
    \end{itemize}
\end{frame}

\begin{frame}{Ví dụ}
    \begin{align}
        \norm{E\left[A-\lambda I,B\right]^{-1}D}&=\sup_{\left|v\right|=1}d\left(0,E\left[A-\lambda I,B\right]^{-1}D\left(v\right)\right) \nonumber\\
        %%
        &=\begin{cases}
        \sup_{\left|v\right|=1}\left\{ \inf\norm{\left[\begin{array}{c}
        x_{1}\\
        x_{2}
        \end{array}\right]}_{\infty}\right\}  & \lambda\ne-1,\\
        \sup_{\left|v\right|=1}\left\{ \inf\left|v\right|\right\}  & \lambda=-1,
        \end{cases} \nonumber\\
        %%
        &=\begin{cases}
        \sup_{\left|v\right|=1}\left\{ \frac{2\left|v\right|}{\left|\lambda-1\right|+1}\right\}  & \lambda\ne-1,\\
        \sup_{\left|v\right|=1}\left\{ \left|v\right|\right\}  & \lambda=-1,
        \end{cases} \nonumber\\
        %%
        &=\begin{cases}
        \frac{2}{\left|\lambda-1\right|+1}, & \lambda\ne-1,\\
        1 & \lambda=-1.
        \end{cases}
    \end{align}
\end{frame}


\begin{frame}{Ví dụ}
    \begin{align}
        \sup_{\lambda\in\C}\norm{E\left[A-\lambda I,B\right]^{-1}D}&=2,\\
        r_{\C}^{D,E}\left(A,B\right)&=\frac{1}{2}.
    \end{align}
\end{frame}





% \subsection{Một số trường hợp cụ thể}
% \begin{frame}{}
% \begin{itemize}
%     \item Hệ
%     \begin{equation}
%         \left(A,B\right)\in\C^{n\times\left(n+m\right)}.
%     \end{equation}
%     \item Bán kính điều khiển
%     \begin{equation}
%         r_{\C}\left(A,B\right)=\frac{1}{\sup_{\lambda\in\C}\norm{W_{\lambda}^{\dagger}}}.
%     \end{equation}
% \end{itemize}    
% \end{frame}

% \begin{frame}{Giả thiết}
%     Giả sử cặp điều khiển được $\left(A,B\right)$ chịu nhiễu có cấu trúc
% theo công thức:
% \begin{equation}
% \left[A,B\right]\leadsto\left[A,B\right]+D\Delta E,\quad\Delta\in\C^{l\times q},\label{eq:3.7}
% \end{equation}
% trong đó $D\in\C^{n\times l}$,$E\in\C^{q\times\left(n+m\right)}.$
% Do $W_{\lambda}W_{\lambda}^{\dagger}Du=Du,\forall u\in\C^{l},$ ta
% có $W_{\lambda}^{\dagger}Du\in W_{\lambda}^{-1}\left(Du\right).$
% Vì vậy $W_{\lambda}^{-1}\left(Du\right)=W_{\lambda}^{\dagger}Du+W_{\lambda}^{-1}\left(0\right)=W_{\lambda}^{\dagger}Du+\ker W_{\lambda}$
% và dẫn đến:
% \begin{equation}
% \left(EW_{\lambda}^{-1}D\right)\left(u\right)=\left(EW_{\lambda}^{-1}\right)\left(Du\right)=EW_{\lambda}^{\dagger}Du+E\ker W_{\lambda}\label{eq:3.8}
% \end{equation}
% \end{frame}

% \begin{frame}{Hệ quả}
%     Giả sử $E^{*}E\ker W_{\lambda}\subset\ker W_{\lambda},\forall\lambda\in\C.$
% Khoảng cách cấu trúc từ cặp điều khiển đươc $\left(A,B\right)$ tới
% trạng thái không điều khiển được cùng với tổ hợp nhiễu có cấu trúc
% của công thức \eqref{eq:3.7} cho bởi công thức sau:
% \begin{equation}
% r_{\C}^{D,E}\left(A,B\right)=\frac{1}{\sup_{\lambda\in\C}\norm{EW_{\lambda}^{\dagger}D}}\label{eq:3.9}
% \end{equation}
% trong đó $W_{\lambda}^{\dagger}=W_{\lambda}^{*}\left(W_{\lambda}W_{\lambda}^{*}\right)^{-1}$
% đại diện nghịch đảo Moore-Penrose của $W_{\lambda}=\left[A-\lambda I,B\right].$
% \end{frame}

% \begin{frame}{Hệ quả}
%     Giả sử cặp điều khiển được $\left(A,B\right)\in\C^{n\times n}\times\C^{n\times m}$
% chịu nhiễu theo công thức \eqref{eq:3.7} trong đó $E\in\C^{q\times\left(m+n\right)}$
% có hạng cột đầy đủ. Khi đó khoảng cách có cấu trúc từ $\left(A,B\right)$
% tới trạng thái không điểu khiển được:
% \begin{equation}
% r_{\C}^{D,E}\left(A,B\right)=\frac{1}{\sup_{\lambda\in\C}\norm{\left(W_{\lambda}\left(E^{*}E\right)^{-1/2}\right)^{\dagger}D}}.\label{eq:3.10}
% \end{equation}
% \end{frame}

% \begin{frame}{Hệ quả}
%     Giả sử cặp điều khiền được $\left(A,B\right)\in\C^{n\times n}\times\C^{n\times m}$
% chịu nhiễu theo công thức \eqref{eq:3.13} trong đó $E_{A}\in\C^{q_{1}\times n}$
% có hạng cột đầy đủ và $\im B^{*}\subset\im E_{B}^{*}$. Khi đó:
% \begin{equation}
% r_{\C}^{D,E}\left(A,B\right)=\inf_{\lambda\in\C}\sigma_{\min}\left(\begin{bmatrix}E_{A}^{*\dagger}\left(A^{*}-\lambda I\right)\\
% E_{B}^{*\dagger}B^{*}
% \end{bmatrix},D^{*}\right),\label{eq:3.14}
% \end{equation}
% trong đó $E_{B}^{*\dagger}$ nghịch đảo tổng quát Moore-Penrose của $E_{B}^{*}$.
% \end{frame}

\section{Đa nhiễu}
% \begin{frame}{Slide nháp}
% Nội dung cần nói
% \begin{itemize}
%     \item Dẫn dắt tại sao lại cần đa nhiễu (không có trong slide)
%     \item Định nghĩa đa nhiễu
%     \item Định nghĩa của bán kính điều khiển
%     \item Đưa ra các theorem về công thức tính bán kính điều khiển
% \end{itemize}
% {\large Ưu tiên đưa các theorem về công thức trước, sau đó đưa các định nghĩa và công thức liên quan vào, vì các bổ đề có thể chỉ dùng để chứng minh chứ không cần thiết cho nội dung định lý}
% \end{frame}
\subsection{Bán kính điều khiển dưới đa nhiễu}

\begin{frame}{Định nghĩa}
    \begin{itemize}
        \item Hệ chịu đa nhiễu
        \begin{align}
            \left[A,B\right]\leadsto\left[\tilde{A},\tilde{B}\right]=\left[A,B\right]+\sum_{i=1}^{N}D_{i}\Delta_{i}E_{i}.
        \end{align}
        \item Nhiễu
        \begin{align}
            \Delta &=\left(\Delta_{1},\ldots,\Delta_{N}\right)\in\Pi_{i=1}^{N}\C^{l_{i}\times q_{i}}\\
            \norm{\Delta} &=\sum_{i=1}^{N}\norm{\Delta_{i}}
        \end{align}
    \end{itemize}
\end{frame}

\begin{frame}{Định nghĩa}
    \begin{itemize}
        \item Bán kính điều khiển
        \begin{align}
            r_{\C}^{\MP}\left(A,B\right)=\inf\left\{ \norm{\Delta}\colon\Delta=\left(\Delta_{i}\right)_{i\in\underline{N}},\left[A,B\right]+\sum_{i=1}^{N}D_{i}\Delta_{i}E_{i}\text{ không điều khiển được}\right\} .
        \end{align}
    \end{itemize}
\end{frame}

% \begin{frame}{Bổ đề}
%     Giả sử $U$ là không gian con của $\C^{k}$ và $0\ne\hat{v}_{0}^{*}\notin U^{\perp}$.
% Khi đó, tồn tại một $0\ne v_{0}^{*}\in\hat{v}_{0}^{*}+U^{\perp}$,
% $0\ne x_{0}\in U$ sao cho
% \[
% \abs{v_{0}^{*}x_{0}}=\norm{v_{0}^{*}}\norm{x_{0}}.
% \]
% \end{frame}

\begin{frame}{Định lý}
    \begin{itemize}
        \item Ký hiệu
        \begin{equation}
            P\preceq Q\iff\norm{Px}\le\norm{Qx},\quad\forall x\in\C^{m}
        \end{equation}
        \item \(H\in\C^{k\times\left(n+m\right)}\), $E_{i}\preceq H\, \forall i\in\underline{N}$
        \begin{equation}
            \left[\max_{i\in\underline{N}}\sup_{\lambda\in \C}\norm{HW_{\lambda}^{-1}D_{i}}\right]^{-1}\le r_{\C}^{\MP}\left(A,B\right)\le\left[\max_{i\in\underline{N}}\sup_{\lambda\in \C}\norm{E_{i}W_{\lambda}^{-1}D_{i}}\right]^{-1}.
        \end{equation}
        \item $E_{i}=\alpha_{i}E_{1}$, $\alpha_{i}\in\C \,\forall i\in\underline{N}$,
        \begin{equation}
            r_{\C}^{\MP}\left(A,B\right)=\left[\max_{i\in\underline{N}}\sup_{\lambda\in \C}\norm{E_{i}W_{\lambda}^{-1}D_{i}}\right]^{-1}.
        \end{equation}
    \end{itemize}
\end{frame}


\subsection{Ví dụ}
\begin{frame}{Ví dụ 2}
    \begin{align}
        \dot{x}= & Ax\left(t\right)+Bu\left(t\right),\\
        A= & \left[\begin{array}{cc}
        0 & 1\\
        -2 & 0
        \end{array}\right],\\
        B= & \left[\begin{array}{c}
        0\\
        1
        \end{array}\right],\\
        \left[\begin{array}{ccc}
        0 & 1 & 0\\
        -2 & 0 & 1
        \end{array}\right]\leadsto & \left[\begin{array}{ccc}
        \delta_{1} & 1+\delta_{2} & 0\\
        -2+\delta_{3} & 0 & 1+\delta_{4}
        \end{array}\right].
    \end{align}
\end{frame}

\begin{frame}{Ví dụ 2}
    \begin{align}
        D_{1} & =\left[\begin{array}{cc}
        1 & 0\\
        0 & 0
        \end{array}\right],D_{2}=\left[\begin{array}{cc}
        0 & 0\\
        0 & 1
        \end{array}\right],\\
        E_{1} & =\left[\begin{array}{ccc}
        1 & 0 & 0\\
        0 & 1 & 0\\
        0 & 0 & 0
        \end{array}\right],E_{2}=\left[\begin{array}{ccc}
        1 & 0 & 0\\
        0 & 0 & 0\\
        0 & 0 & 1
        \end{array}\right],\\
        H & =\left[\begin{array}{ccc}
        1 & 0 & 0\\
        0 & 1 & 0\\
        0 & 0 & 1
        \end{array}\right].
    \end{align}
\end{frame}


\begin{frame}{Ví dụ 2}
    \begin{align}
        E_{1}\left[A-\lambda I,B\right]^{-1}D_{1}\left[\begin{array}{c}
        x\\
        y
        \end{array}\right] & =E_{1}\left[A-\lambda I,B\right]^{-1}\left[\begin{array}{c}
        x\\
        0
        \end{array}\right] \nonumber\\
         & =\left\{ \left[\begin{array}{c}
        p\\
        x+\lambda p\\
        0
        \end{array}\right]:p\in\C\right\} ,\\
        E_{2}\left[A-\lambda I,B\right]^{-1}D_{2}\left[\begin{array}{c}
        x\\
        y
        \end{array}\right] & =E_{2}\left[A-\lambda I,B\right]^{-1}\left[\begin{array}{c}
        0\\
        y
        \end{array}\right] \nonumber\\
         & =\left\{ \left[\begin{array}{c}
        p\\
        0\\
        y+\left(\lambda^{2}+2\right)p
        \end{array}\right]:p\in\C\right\} .
    \end{align}
\end{frame}

\begin{frame}{Ví dụ 2}
    \begin{align}
    \norm{E_{1}\left[A-\lambda I,B\right]^{-1}D_{1}} & =\frac{1}{1+\left|\lambda\right|},\\
    \norm{E_{2}\left[A-\lambda I,B\right]^{-1}D_{2}} & =\frac{1}{1+\left|\lambda^{2}+2\right|}.
    \end{align}
\end{frame}

\begin{frame}{Ví dụ 2}
    \begin{align}
        H\left[A-\lambda I,B\right]^{-1}D_{1}\left[\begin{array}{c}
        x\\
        y
        \end{array}\right] & =\left[A-\lambda I,B\right]^{-1}\left[\begin{array}{c}
        x\\
        0
        \end{array}\right] \nonumber\\
         & =\left\{ \left[\begin{array}{c}
        p\\
        x+\lambda p\\
        \lambda x+\left(\lambda^{2}+2\right)p
        \end{array}\right]:p\in\C\right\} ,\\
        H\left[A-\lambda I,B\right]^{-1}D_{2}\left[\begin{array}{c}
        x\\
        y
        \end{array}\right] & =\left[A-\lambda I,B\right]^{-1}\left[\begin{array}{c}
        0\\
        y
        \end{array}\right] \nonumber\\
         & =\left\{ \left[\begin{array}{c}
        p\\
        \lambda p\\
        y+\left(\lambda^{2}+2\right)p
        \end{array}\right]:p\in\C\right\} .
    \end{align}
\end{frame}



\begin{frame}{Ví dụ 2}
\begin{itemize}
    \item Chọn $p=0$ và $p=-\frac{x}{\lambda}$, $\forall\lambda\in\C$:
    \begin{align}
        d\left(0,H\left[A-\lambda I,B\right]^{-1}D_{1}\left[\begin{array}{c}
        x\\
        y
        \end{array}\right]\right) & \le\min\left\{ \max\left\{ \left|x\right|,\left|\lambda x\right|\right\} ,\left|\frac{2x}{\lambda}\right|\right\} \le\sqrt{2}\left|x\right|.
    \end{align}
    
    % \item 
    \begin{align}
        \implies\norm{H\left[A-\lambda I,B\right]^{-1}D_{1}}\le\sqrt{2}\quad\forall\lambda\in\C.
    \end{align}
    \item Dấu bằng với $\lambda=\sqrt{2}i$:
    \begin{align}
        \sup_{\lambda\in\C}\norm{H\left[A-\lambda I,B\right]^{-1}D_{1}}=\sqrt{2}.
    \end{align}
\end{itemize}
\end{frame}

\begin{frame}{Ví dụ 2}
\begin{itemize}
    \item Tương tự, với $p=0$:
    \begin{align}
        d\left(0,H\left[A-\lambda I,B\right]^{-1}D_{2}\left[\begin{array}{c}
        x\\
        y
        \end{array}\right]\right) & \le\left|y\right|,\\
        \norm{H\left[A-\lambda I,B\right]^{-1}D_{2}\left[\begin{array}{c}
        x\\
        y
        \end{array}\right]} & \le1\quad\forall\lambda\in\C.
    \end{align}
    \item Dấu bằng khi $\lambda=\sqrt{2}i$:
    \begin{align}
        \sup_{\lambda\in\C}\norm{H\left[A-\lambda I,B\right]^{-1}D_{2}}=1.
    \end{align}
\end{itemize}
\end{frame}

\begin{frame}{Ví dụ 2}
    \begin{itemize}
        \item Ước lượng bán kính điều khiển
        \begin{align}
        \frac{1}{\sqrt{2}}=\frac{1}{\max\left\{ 1,\sqrt{2}\right\} }\le r_{\C}^{\MP}\left(A,B\right)\le\frac{1}{\max\left\{ \sup\frac{1}{1+\left|\lambda\right|},\sup\frac{1}{1+\left|\lambda^{2}+2\right|}\right\} }=1.
        \end{align}
    \end{itemize}
\end{frame}
\input{parts/multi-perturbation-example.tex}

\section{Tài liệu tham khảo}
\begin{frame}{Tài liệu tham khảo}
  \nocite{*}
  \bibliographystyle{plainnat}
  \bibliography{example.refs.bib}
\end{frame}

\begin{frame}{~}
  \Large \color{hustred}{Cảm ơn mọi người đã chú ý lắng nghe!}
\end{frame}

\end{document}
