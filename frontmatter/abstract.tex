\abstractpage{
  This thesis presents a web application to serve \acrfull{gec} systems (from now on refers as GecWeb) so that they can be easily used by the general public.
  The main design goals is to make GecWeb accessible to as many users as possible, including users who have a slow Internet connection and who use mobile phones as their main devices to connect to the Internet.
  GecWeb provides three state-of-the-art base GEC systems using sequence tagging, as well as two state-of-the-art GEC system combination methods using two approaches (edit-based and text-based).

  % It is suggested to have the abstract in both language (Vietnamese and English).
  \newpage
  \begin{center}
    \vspace*{1pt}
    \Large \textcolor{Crimson}{\textit{Ứng dụng web sửa lỗi ngữ pháp sử dụng học máy}} \normalsize\\
    \vspace*{15pt}
    {\bf Tóm tắt đồ án} \rm
  \end{center}

  Đồ án này trình bày một ứng dụng web để phục vụ chỉnh sửa lỗi ngữ pháp (sau đây gọi là GecWeb), giúp công chúng dễ dàng tiếp cận và sử dụng.
  GecWeb được tập trung thiết kế nhằm đảm bảo khả năng tiếp cận rộng rãi nhất có thể, bao gồm cả những người dùng có kết nối Internet chậm và những người sử dụng điện thoại di động làm thiết bị chính để truy cập Internet.
  GecWeb cung cấp ba hệ thống chỉnh sửa lỗi ngữ pháp (GEC) tiên tiến sử dụng phương pháp gán nhãn chuỗi (sequence tagging), cũng như hai phương pháp kết hợp hệ thống GEC hiện đại dựa trên hai cách tiếp cận khác nhau (dựa trên chỉnh sửa và dựa trên văn bản).
}
