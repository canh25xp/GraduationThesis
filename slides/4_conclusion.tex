\section{Conclusion}

\subsection{Achievements}
\begin{frame}{Achievements}
  \begin{itemize}
    \item A web-based application for GEC that can be easily access by general public.
    \item Designed with accessibility in mind.
    \item Flexible to host different type of GEC models and system combination methods.
    \item Designed to be modular and extensible.
  \end{itemize}
\end{frame}

\note{
  In conclusion, this thesis has presented GecWeb, a web-based application for GEC that can be easily access by general public.

  GecWeb is designed with accessibility in mind, make it easier to use on both desktop and mobile devices.

  GecWeb is flexible to host different type of GEC models and system combination methods selected by the user.

  Additionally, GecWeb is designed to be modular and extensible, so that it can be easily extended to support more GEC models and system combination methods.
}

\subsection{Future work}
\begin{frame}{Future work}
  \begin{itemize}
    \item Deploy the system to a more realiable server.
    \item Scale the system to handle more users.
  \end{itemize}
\end{frame}

\note{
  Although archived the initial goal of creating a web-based application for GEC, GecWeb is no where near production ready.

  The backend is currently running on a low-end server, which is not enough to run all 3 models and prune to crash(which it sometimes did).

  I did not take care of the security aspect of the application, such as input validation, output encoding, et
  cetera.

  As well as the performance aspect, such as caching, load balancing, et cetera.
}
