\section{Introduction}

\subsection{Motivation}

\begin{frame}{Motivation}
  \begin{figure}
    \begin{center}
      \includegraphics[width=\textwidth]{figures/ef-epi-2024-english-crop.pdf}
      \begin{textblock*}{8cm}(\paperwidth-9cm, \paperheight-2.5cm)  % (x,y) coordinates from top-left
        \textbf{\Large $> 1.4$ billion speakers}

        \textbf{\Large $\sim 75\%$ non-native}~\cite{Ethnologue-2024}
      \end{textblock*}
    \end{center}
    \caption{English Proficiency bands by countries~\cite{ef-epi-2024}}
  \end{figure}
\end{frame}

\note{
  English is one of the most widely used languages globally, spoken by more than 1.4 billion speakers, with almost 75\% of them being non-native speakers

  As the number of english-as-a-second-language (esl) and english-as-a-foreign-language (efl) learners continues to grow, the need for an effective language learning tools has increased significantly.

  However, grammatical and spelling errors remain common challenges for many writers, affecting clarity and professionalism.
}

\subsection{Definition}

\begin{frame}{Definition of Grammatical Error Correction}
  \centering
  {\Huge
    \textcolor{red}{G}rammatical
    \textcolor{red}{E}rror
    \textcolor{red}{C}orrection
  }

  \vfill

  {\Large
    He \textcolor{red}{go} to store and \textcolor{red}{buyed} some \textcolor{red}{apple}.
  }

  \noindent\rule[0.5ex]{\linewidth}{1pt}

  {\Large
    He \textcolor{green}{goes} to \textcolor{green}{the} store and \textcolor{green}{bought} some \textcolor{green}{apples}.
  }

  \vfill
\end{frame}

\note[itemize]{
  \item Grammatical Error Correction, or GEC, is the task of automatically detecting and correcting errors in text.

  \item Despite its name, the task is not limited to grammatical errors, such as missing prepositions and mismatched subject-verb agreement but it also expect to fix orthographic and semantic errors, such as misspellings and word choice errors.

  \item The term Grammatical Error Correction is thus a bit misleading but is nevertheless now commonly understood by the research community.

  \item A more appropriate term would be Language Error Correction.
}

\subsection{Usage barriers}

\begin{frame}{Usage barriers}

  \begin{columns}
    \begin{column}{0.4\textwidth}
      \begin{itemize}
        \item Command-line interface
        \item Requires capable hardware
      \end{itemize}
    \end{column}

    \begin{column}{0.6\textwidth}
      \begin{figure}
        \centering
        \includegraphics[width=0.8\textwidth]{cli}
        \caption{Gector command line interface usage}
      \end{figure}
    \end{column}
  \end{columns}
\end{frame}

\note[itemize]{
  \item Despite significant recent advancements in GEC technology, many state of the art systems remain inaccessible to the general public due to their reliance on command line interfaces and high performance computing resources.

  \item This creates a barrier for non technical users, particularly those in developing countries with limited access to advanced technology and slow internet connections.
}

\begin{frame}{Usage barriers}
  \begin{itemize}
    \item Command-line interface
    \item Requires capable hardware
  \end{itemize}

  \vspace{1cm}

  {\Large $\Rightarrow$ Solution: A lightweight web interface}
\end{frame}

\note{
  This is where GecWeb comes in, it provides a lightweight, user friendly GEC system that can be easily accessed via mobile devices or low-end computers.
}
