\chapter{Introduction}

\section{Motivation}
\label{section:motivation}

English is one of the most widely used languages globally, serving as a common medium of communication for over 1.4 billion people worldwide, with almost 75\% of them being non-native speakers~\citep{eberhard2015ethnologue}.
As the number of \acrfull{esl} and \acrfull{efl} learners continues to grow, the demand for effective language learning tools and resources has increased significantly.
However, grammatical and spelling errors remain common challenges for many writers, affecting clarity and professionalism.

\acrfull{gec} is a task that aims to automatically detect and correct errors that are present in a text, including grammatical errors, orthographic errors, misspellings, word choice errors, etc. \citep{ng-etal-2014-conll}

Despite significant recent advancements in GEC technology, many \acrfull{sota} systems remain inaccessible to the general public due to their reliance on command-line interfaces and high-performance computing resources.
This creates a barrier for non-technical users, particularly those in developing countries with limited access to advanced technology and slow internet connections.
The need for a lightweight, user-friendly GEC system that can be easily accessed via mobile devices is therefore urgent.

The development of such a system would benefit not only ESL and EFL learners but also native speakers who occasionally make mistakes.
Additionally, improved GEC tools can enhance the quality of other \acrfull{nlp} tasks, such as machine translation and speech recognition, thereby contributing to broader advancements in the field of NLP.

\section{Objectives and scope of the graduation thesis}
\label{section:objective}
Given the observations mention in~\ref{section:motivation}, this thesis aims to address the following key challenges and limitations:

\begin{itemize}
  \item Accessibility: Many existing GEC systems require significant computing resources, making them difficult to use on mobile devices or low-speed internet connections.
  \item Correction Accuracy: Individual GEC models often struggle with certain grammatical structures, necessitating the need for model combination techniques to enhance performance.
  \item Usability and Deployment: A practical GEC solution should provide a lightweight, user-friendly web interface that ensures smooth and efficient interaction for a diverse user base.
\end{itemize}

\section{Tentative solution}
\label{section:solution}

\section{Thesis organization}
\label{section:organization}
