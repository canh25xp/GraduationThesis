\chapter{Introduction}
\label{chapter:introduction}

\section{Motivation}
\label{section:motivation}

English is one of the most widely used languages globally, serving as a common medium of communication for over 1.4 billion people worldwide, with almost 75\% of them being non-native speakers~\citep{eberhard2015ethnologue}.
As the number of \acrfull{esl} and \acrfull{efl} learners continues to grow, the demand for effective language learning tools and resources has increased significantly.
However, grammatical and spelling errors remain common challenges for many writers, affecting clarity and professionalism.

\acrfull{gec} is a task that aims to automatically detect and correct errors that are present in a text, including grammatical errors, orthographic errors, misspellings, word choice errors, etc. \citep{ng-etal-2014-conll}

Despite significant recent advancements in GEC technology, many \acrfull{sota} systems remain inaccessible to the general public due to their reliance on command line interfaces and high performance computing resources.
This creates a barrier for non technical users, particularly those in developing countries with limited access to advanced technology and slow internet connections.
The need for a lightweight, user friendly GEC system that can be easily accessed via mobile devices is therefore urgent.

The development of such a system would benefit not only \acrshort{esl} and \acrshort{efl} learners but also native speakers who occasionally make mistakes.
Additionally, improved GEC tools can enhance the quality of other \acrfull{nlp} tasks, such as machine translation and speech recognition, thereby contributing to broader advancements in the field of NLP.

\section{Objectives and scope of the graduation thesis}
\label{section:objective}

Currently, there are several web services, such as those provided by Grammarly and John Snow Labs, offer ready-to-use English text correction.
However, these services are not open-source, limiting their adaptability for deploying different GEC systems.
Therefore, I will only compare and evaluate some available open-source English correction tools, noticeably GECko+ and MiSS.

GECko+ is an English language assisting tool that corrects mistakes of various types in written texts.
It combines a sentence level GEC model, GECToR XLNet, and a sentence ordering model.
When a user inputs a text into the system, it segments the text into sentences and corrects the sentences with GECToR before re-ordering them by the sentence ordering model.
However, GECko+ lacks the options of choosing the GEC base models and using
system combination methods.
It is also unclear how easy it is to extend GECko+ to other GEC systems.

MiSS, on the other hand, is a comprehensive tool for machine translation that includes grammatical error correction as a feature.
The main machine translation features of MiSS include basic machine translation, simultaneous machine translation, and back translation for quality evaluation.
For the GEC part, it uses GECToR XLNet for English GEC and GECToR with BERT-Chinese and BERT-Japanese models for Chinese and Japanese GEC, respectively.
Like GECko+, MiSS also lacks the options of choosing the GEC base models and using system combination methods.

Based on the above analysis, this thesis aims to develop GecWeb (Grammatical Error Correction Web), a web-based application designed to make state-of-the-art GEC systems more accessible to the general public.
GecWeb addresses the limitations of existing GEC tools-such as their reliance on command-line interfaces, lack of mobile support, and limited customization-by offering a lightweight, user-friendly interface.
This application is specifically designed to function efficiently across different screen sizes and varying internet speeds, making it particularly beneficial for users in developing countries.

\section{Tentative solution}
\label{section:tentative}

My proposed solution involves the development of a web application that leverage state-of-the-art GEC models (GECToR-Roberta, GECToR-XlNet and GECToR-Bert) and combination methods, namely ESC (Edit-based System Combination) and MEMT (Multi-Engine Machine Translation).

The front end of GecWeb is built using Flask and Bootstrap, providing a lightweight yet responsive web interface.
Flask was chosen for its simplicity and seamless integration with the back end, while Bootstrap ensures a modern and mobile-friendly user experience.
The back-end is implemented using Flask-RESTful API, which efficiently handles client requests and provides structured API endpoints.
Flask-RESTful was selected due to its minimal overhead, ease of use, and flexibility in designing scalable web services.
The core grammatical error correction functionality is powered by GECToR, a transformer-based machine learning model.
GECToR was chosen because of its state-of-the-art performance in handling complex grammatical errors while maintaining high accuracy and efficiency.

The main contribution of this thesis is the creation of a lightweight, modular GEC system that can be easily extended to include new models and combination methods.
The system will be designed to minimize data transfer overhead, making it suitable for users with slow internet connections.
Furthermore, the system will feature a responsive web interface that adapts to different screen sizes, ensuring a seamless user experience on both desktop and mobile devices.

\section{Thesis organization}
\label{section:organization}

The rest of this graduation thesis is organized as follows.

Chapter~\ref{chapter:requirement} focuses on presenting a detailed survey of the current state of GEC systems, including an analysis of user needs and existing products.
This chapter will also outline the functional and non-functional requirements for GecWeb, based on the identified limitations of current systems.

Chapter~\ref{chapter:methodology} introduces the methodologies and technologies used in the development of GecWeb.
This chapter will provide an overview of the sequence tagging approaches, as well as the combination methods employed in the system.
The chapter will also discuss the rationale behind the choice of technologies and their relevance to the requirements outlined in Chapter~\ref{chapter:requirement}.

Chapter~\ref{chapter:design} discusses in detail the design, implementation, and evaluation of GecWeb.
This chapter will cover the system's architecture, user interface design, and database design.
It will also describe the tools and libraries used in the development process, as well as the testing and deployment of the application.

Chapter~\ref{chapter:contribution} presents the solutions and contributions of this thesis, focusing on the innovative aspects of GecWeb and the challenges overcome during its development.
This chapter will highlight the system's modularity, lightweight design, and ability to support multiple GEC models and combination methods.

Finally, Chapter~\ref{chapter:conclusion} concludes the thesis by summarizing the achievements of GecWeb and discussing potential future work.
This chapter will also provide an analysis of the system's performance compared to existing GEC tools and suggest directions for further improvement.
