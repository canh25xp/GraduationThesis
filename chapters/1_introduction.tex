\chapter{Introduction}

\section{Motivation}
\label{section:motivation}

English is one of the most widely used languages globally, serving as a common medium of communication for over 1.4 billion people worldwide, with almost 75\% of them being non-native speakers \citep{eberhard2015ethnologue}.
As the number of \acrfull{esl} and \acrfull{efl} learners continues to grow, the demand for effective language learning tools and resources has increased significantly.
However, grammatical and spelling errors remain common challenges for many writers, affecting clarity and professionalism.

\acrfull{gec} is a task that aims to automatically detect and correct errors that are present in a text, including grammatical errors, orthographic errors, misspellings, word choice errors, etc. \citep{ng-etal-2014-conll}
Traditional grammar checking tools often rely on rule based approaches, which may lack flexibility and adaptability to diverse writing styles.
With recent advancements in machine learning and \acrfull{nlp}, modern \acrshort{gec} systems have significantly improved in accuracy and efficiency.

This thesis presents GecWeb, a grammatical error correction web application designed to provide an accessible and efficient grammatical error correction system for the general public.
Unlike computationally expensive sequence-to-sequence models, GecWeb utilizes a sequence tagging approach, which enables faster and more resource-efficient corrections.
Additionally, it integrates multiple state-of-the-art GEC models and system combination methods to enhance correction quality.

A key objective of GecWeb is to ensure usability across a wide range of users, including those with slow internet connections or who primarily access the web through mobile devices.
By focusing on performance optimization and user accessibility, this application aims to bridge the gap between cutting-edge NLP research and practical real-world usage.

% This thesis explores the design, implementation, and evaluation of GecWeb, highlighting its effectiveness compared to existing GEC solutions.
% The research also investigates different system combination approaches—edit-based and text-based—to improve correction accuracy.
% Through rigorous testing and analysis, this work demonstrates the potential of machine learning-based GEC systems in providing accurate, efficient, and accessible grammatical error correction for users worldwide.

\section{Objectives and scope of the graduation thesis}
\label{section:objective}
Given the observations mention in~\ref{section:motivation}, this thesis aims to address the following key challenges and limitations:

\begin{itemize}
  \item Accessibility: Many existing GEC systems require significant computing resources, making them difficult to use on mobile devices or low-speed internet connections.
  \item Correction Accuracy: Individual GEC models often struggle with certain grammatical structures, necessitating the need for model combination techniques to enhance performance.
  \item Usability and Deployment: A practical GEC solution should provide a lightweight, user-friendly web interface that ensures smooth and efficient interaction for a diverse user base.
\end{itemize}

\section{Tentative solution}
\label{section:solution}

\section{Thesis organization}
\label{section:organization}
