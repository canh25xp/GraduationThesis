\chapter{Solution and contribution}
\label{chapter:contribution}

The development of GecWeb presented a series of challenges that required innovative solutions and careful consideration of both technical and user-centric requirements.
One of the most significant contributions of this thesis is the creation of a lightweight, modular, and user-friendly web application that integrates state-of-the-art grammatical error correction (GEC) models and system combination methods.
This chapter highlights the key contributions and solutions developed during the thesis, focusing on the challenges overcome and the novel approaches implemented to achieve the project's objectives.

One of the primary challenges was designing a system that could efficiently handle multiple GEC models while maintaining a lightweight and responsive user interface.
To address this, I adopted a three-tier architecture, separating the presentation layer (user interface), application layer (API and logic), and data layer (GEC models and combination methods).
This modular design allowed for independent development and scaling of each component, ensuring that the system could be easily extended with new models or combination methods in the future.
The separation of the front-end and back-end also enabled the system to leverage GPU resources for computationally intensive tasks while keeping the user interface lightweight and accessible on devices with limited processing power.

Another significant contribution was the integration of system combination methods, specifically Edit-based System Combination (ESC) and Multi-Engine Machine Translation (MEMT), to enhance the accuracy of grammatical error corrections.
Implementing these methods required careful consideration of how to align and merge the outputs of multiple GEC models.
For ESC, I adapted the original code to work with in-memory data rather than file-based inputs, significantly improving processing speed and reducing overhead.
For MEMT, I implemented a beam search algorithm to generate and score candidate sentences based on features such as language model scores and n-gram similarity.
These combination methods not only improved the quality of corrections but also provided users with the flexibility to choose between different approaches based on their specific needs.

The user interface design was another area where significant effort was invested to ensure accessibility and usability.
By leveraging Bootstrap, I created a responsive and intuitive interface that adapts to various screen sizes, from mobile phones to desktop monitors.
The inclusion of a highlight feature, which visually marks corrections in the output text, was particularly innovative.
This feature not only makes it easier for users to identify changes but also provides explanations for each correction, helping language learners understand their mistakes.
The implementation of this feature required integrating ERRANT, a tool for error annotation, to parse and compare input and output sentences, extracting detailed information about the types of corrections made.

Performance optimization was another critical aspect of the project.
Given the target audience of users in developing countries with potentially slow internet connections, it was essential to minimize data transfer overhead and ensure fast processing times.
By hosting the GEC models on a GPU-powered server and using efficient batching and tokenization techniques, I achieved correction speeds of up to 723 words per second for the GECToR-Roberta model.
Additionally, the use of Flask-RESTful API ensured that the system could handle multiple user requests simultaneously without significant latency.
These optimizations made GecWeb suitable for real-time use, even in resource-constrained environments.

The development process also involved rigorous testing and validation to ensure the system's reliability and correctness.
I implemented a comprehensive testing framework using pytest, which included unit tests for individual components and regression tests to verify the consistency of the system's output.
Automated testing through GitHub Actions further streamlined the development process, ensuring that any changes to the codebase did not introduce unintended side effects.
Manual API testing using curl provided additional validation, confirming that the system functioned as expected in real-world scenarios.

Finally, the deployment strategy was designed to maximize accessibility and scalability.
By hosting the Gec API on Hugging Face Inference Endpoints, I leveraged GPU resources to ensure fast and efficient text processing.
The Gec Web interface, hosted on Hugging Face Spaces, provided a seamless and user-friendly experience accessible from any modern browser.
This cloud-based approach not only simplified deployment and maintenance but also made the system widely accessible to users worldwide.

In conclusion, the development of GecWeb represents a significant contribution to the field of grammatical error correction, particularly in making state-of-the-art GEC systems more accessible to non-technical users.
The modular architecture, integration of advanced combination methods, responsive user interface, and performance optimizations collectively address the limitations of existing GEC tools.
By overcoming these challenges, GecWeb provides a robust and scalable solution that benefits ESL and EFL learners, native speakers, and NLP researchers alike.
The lessons learned and solutions developed during this thesis lay the groundwork for future advancements in GEC technology and its applications.
